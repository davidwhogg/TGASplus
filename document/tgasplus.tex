% This file is part of the TGASplus project.
% Copyright 2016 the authors.

% ## Style notes
% - Use n-dash (--) for color--magnitude diagram.

\documentclass[12pt]{article}
\usepackage{url}

% text definitions
\include{gitstuff}
\newcommand{\project}[1]{\textsl{#1}}
\newcommand{\acronym}[1]{\small{#1}}
\newcommand{\gaia}{\project{Gaia}}
\newcommand{\tgas}{\project{\acronym{TGAS}}}

% math definitions
\newcommand{\given}{\,|\,}

% style stuff
\urlstyle{sf}
\frenchspacing
\sloppy\sloppypar\raggedbottom
\thispagestyle{empty}

\begin{document}

\section*{Far, far better distances for Gaia stars}

\noindent
DWH, DSTN, JB, others?

\paragraph{Abstract:}
% Context:
Given that the color--magnitude diagram of stars is informative, and
given that stars are drawn from a quasi-stationary distribution, the
best measure of a \gaia\ \tgas\ Catalog star's distance will not be based on
it's tabulated parallax measurement alone (or even at all):
The color and magnitude are both very informative.
Furthermore, because of statistical shrinkage---even if you
only care about a single star in the Catalog---the colors, magnitudes,
and parallaxes of all the \emph{other} stars in the Catalog are \emph{also}
informative.
% Aims:
Here we make use of both of these pieces of information, but no
physical models of stars whatsoever, to greatly improve our distance
estimates for \gaia\ \tgas\ stars.
% Methods:
We transform the color--magnitude diagram to a space in which the
measurement uncertainties are expected to be close to Gaussian in
form.
This permits us to build a numerically tractable hierarchical model
with the Extreme Deconvolution method for modeling distributions of
objects given noisy, heteroskedastic measurements.
The output of the hierarchical model is (among other things) posterior
pdfs for the distance to every star in the Catalog.
% Results:
The posterior pdfs for distance for typical \tgas\ stars is more
informative (more precise) by a factor of XXX relative to any
comparable analysis of the star's parallax measurement alone.
Everything we do here can be done, only far better, for subsequent
\gaia\ data releases.

\section{Philosophy of this project}

\section{Assumptions and method}

\begin{eqnarray}
  d_{kt} &\sim& p(d_k\given\varpi_k,\theta_0)
  \\
  p(d_k\given\theta_0) &=& d^2\,\exp(-\frac{d^2}{L^2})
\end{eqnarray}

\section{Results and discussion}
We ran on everything and got XXX.

Here are some plots comparing to CBJ-style inferences.

HOGG: How much better when we get BpRp and RVS!

All the data used in this project are public. All the code used in
this project is open-source. The main codebase is licensed under the
\project{\acronym{MIT} License} and available at
\url{https://github.com/davidwhogg/TGASplus}. This version of the
manuscript and figures was made at git hash
\textsf{\githash\,(\gitdate)}.

\paragraph{Acknowledgements:}
It is a pleasure to thank various people.
We are funded by various agencies.

\end{document}
